%% LyX 2.2.1 created this file.  For more info, see http://www.lyx.org/.
%% Do not edit unless you really know what you are doing.
\documentclass[english,brazil,oldfontcommands]{abntex2}
\usepackage[T1]{fontenc}
\usepackage[utf8]{inputenc}
\pagestyle{plain}
\usepackage[authoryear]{natbib}
\usepackage{nomencl}
% the following is useful when we have the old nomencl.sty package
\providecommand{\printnomenclature}{\printglossary}
\providecommand{\makenomenclature}{\makeglossary}
\makenomenclature
\usepackage{babel}
\begin{document}
Em Biomedicina, especificamente em radiologia e oncologia, avaliar
a resposta ao tratamento do câncer depende criticamente dos resultados
da análise de imagens pelos especialistas. No entanto, as informações
obtidas dessa análise não são facilmente interpretadas por máquinas. 

Pode-se ver que há uma falta de conexão entre a informação visual
da imagem e sua interpretação. É por essa razão que as tecnologias,
como a web semântica, geram um interesse crescente para aplicação
em biomedicina. Essas tecnologias podem tornar dados em biomedicina
explícitos e computáveis. A comunidade biomédica está em busca de
ferramentas para ajudar com o acesso, consulta e análise da vasta
quantidade de dados gerados pelos avanços no uso da tecnologia na
medicina \citep{marquet2007}.

O uso de imagens na medicina, especificamente no tratamento de pacientes
com câncer, gera enormes quantidades de informações do tipo não-texto.
No entanto, o uso dessas imagens médicas nas tarefas clínicas é importante,
pois permite aos especialistas, diagnosticar, planejar e acompanhar
os pacientes \citep{Levy2009}. Desse modo, um número considerável
de aplicações informáticas, voltadas para área medica, têm sido desenvolvidas.
Grande parte dessas aplicações estão focadas na extração de características
visuais com a ajuda de algoritmos de processamento de imagem.

Embora esses algoritmos possam auxiliar a comunidade biomédica, quanto
ao uso de imagens no tratamento de câncer, eles apresentam problemas
quando uma consulta abstrata e ambígua é feita no contexto da classificação
de pacientes com câncer. Por exemplo, quando um oncologista quer saber
se um tumor já se encontra em estado avançado, perto de se espalhar
para alguma região próxima da origem do câncer, mas não para outras
partes do corpo \citep{Wennerberg2011155}. Portanto, podem haver
algumas dificuldades em manipular interpretações abstratas de imagens,
porque a informação semântica dos laudos sobre a imagem não é considerada
nesses algoritmos. 

\section{Contextualização }

Embora imagens médicas forneçam uma quantidade relevante de informações
para os médicos, estas informações não podem ser integradas facilmente
em aplicações médicas avançadas tais como, sistemas de apoio à decisão
clínica para tratar pacientes com câncer. Especificamente quando os
médicos vão avaliar o progresso individual de um paciente para decidir
sobre novas medidas de tratamento\citep{Zillner2013}. 

No fluxo de trabalho para avaliar o progresso individual de um paciente
com câncer, o radiologista identifica as lesões cancerígenas por meio
de imagens e grava as medições detalhadas sobre as lesões utilizando
anotações. Em seguida, o oncologista analisa e extrai as informações
sobre a localização e o tamanho das lesões tumorais a partir das anotações
feitas pelo radiologista e registra as informações em uma folha de
fluxo. 

As informações, contidas nesta folha de fluxo, são então utilizadas
para os cálculos da taxa de resposta individual de um paciente. A
execução dessa análise final em um único paciente não representa um
problema, mas quando o número de pacientes aumenta essa tarefa se
torna laboriosa e propensa a erros.

A análise dessa folha de fluxo baseia-se no conhecimento do estágio
do câncer (\emph{cancer staging}) do paciente. O estágio do tumor
é um processo de classificação baseado em características como localização
e tamanho do tumor sobre o corpo. Portanto, obter informações sobre
o estágio do câncer é importante para identificar as possíveis opções
de tratamento adequadas. Esse processo de classificação poderia ser
automatizado, a fim de otimizar o trabalho dos médicos que pode se
tornar pesado e propenso a erros, quando o número de pacientes é consideravelmente
grande \citep{Zillner2010}.

Há poucas ferramentas que permitem que radiologistas capturem facilmente
informações semânticas estruturadas como parte do seu fluxo de trabalho
de investigação de rotina \citep{Serique2012}. O projeto \foreignlanguage{english}{Annotation
and Image Markup} (AIM\nomenclature{AIM}{Annotation and Image Markup Project})
desenvolvido pelo \foreignlanguage{english}{cancer Biomedical Informatics
Grid} (caBIG\nomenclature{caBIG}{cancer Biomedical Informatics Grid}),
uma iniciativa do \foreignlanguage{english}{National Cancer Institute}
(NCI\nomenclature{NCI}{National Cancer Institute})\citep{Rubin2008},
fornece um esquema XML\nomenclature{XML}{ eXtensible Markup Language}
para descrever a estruturaanatômica e observações visuais em imagens
utilizando a ontologia RadLex \citep{Kundu2009}. Ele permite uma
representação, armazenamento e transferência consistentes de significados
semânticos sobre imagens. Ferramentas, como por exemplo o ePAD \citep{Rubin2014},
usam o formato AIM. O ePAD permite aos pesquisadores e clínicos criar
anotações semânticas em imagens radiológicas. Ferramentas, como o
ePAD, podem ajudar a reduzir o esforço de se recolher informação semântica
estruturada sobre imagens. No entanto é necessário fazer inferências
sobre essas informacões (lesões de câncer) usando as relações biológicas
e fisiológicas entre as anotações.

\section{Motivação}

O processo de classificação dos pacientes com câncer, através da análise
de imagens, é uma tarefa executada por especialistas, como oncologistas
e radiologistas, com base na inspeção de imagens e pode ser, muitas
vezes, um trabalho intensivo que exige precisão na interpretação das
lesões de câncer. A precisão do especialista é obtida através de formação
e experiência \citep{Depeursinge2014}, mas mesmo com boa formação
e experiência podem ocorrer variações na interpretação de imagens
entre especialistas. Nesse contexto, o desenvolvimento de um sistema
de classificação automática representa uma forte necessidade médica
que pode ajudar a se obter uma maior taxa de precisão na interpretação.

Por outro lado, embora sistemas como o ePAD permitam a criação de
anotações de imagem no formato AIM, facilitando assim a manipulação
e consulta de metadados de imagens em AIM, eles não permitem representar
anotações em um formato que seja diretamente adequado para o raciocínio
(\emph{reasoning}). O AIM fornece apenas um formato para transferência
e armazenamento de dados. Um fato importante é que, atualmente, não
há muitos métodos de \emph{reasoning} semântico para se fazer inferências
sobre lesões cancerígenas utilizando anotações sobre imagens médicas
codificadas pelo AIM \citep{Levy2009}. Outros sistemas, tais como
mint Lesion (MintMedical GmbH, Dossenheim, Germany) and syngo.via
(Siemens Healthcare, Malvern) são softwares proprietários, ou seja,
todos os dados sobre imagens que estão armazenados internamente estão
em um formato proprietário que não pode ser acessado por terceiros
\citep{Rubin2014}.

Podemos ver então que existe uma carência de métodos de \emph{reasonig}
semântico para fazer inferências sobre lesões cancerígenas a partir
de anotações semânticas sobre imagens, baseadas no meios padrão (como
AIM) de adição de informação e conhecimento para uma imagem. Sendo
assim, a principal motivação deste trabalho é a possibilidade de desenvolver
essos métodos de \emph{reasoning} baseados em sistemas de notação,
como\emph{ }\foreignlanguage{english}{\emph{TNM Classification of
Malignant Tumors}} (TNM\nomenclature{TNM}{TNM Classification of Malignant Tumors}).
A fim de classificar automáticamente pacientes com câncer, a partir
de anotações AIM sobre imagens desses pacientes. E finalmente incorporar
esses métodos aos requisitos atendidos por sistemas de anotação de
imagens, como ePAD.

\section{Objetivo}

Este trabalho tem como objetivo gerar automaticamente o\emph{ }\foreignlanguage{english}{\emph{Cancer
Staging}}(estágios do câncer) de lesões cancerígenas presentes em
imagens médicas, utilizando tecnologias de \emph{reasoning} e web
semântica para processar metadados sobre anotações dessas imagens
gerados por ferramentas que utilizan um esquema XML como AIM(ePAD),
para descrever a estrutura anatômica e observações visuais em imagens.
A fim de fornecer aos médicos uma segunda opinião sobre a classificação
dos pacientes com um determinado tipo de câncer e obter uma maior
taxa de precisão na interpretação das lesões. Vale a pena mencionar
que o trabalho atual concentra-se em \emph{staging} do câncer de figado,
devido à disponibilidade de dados.

Para atingir o objetivo proposto, são necessários os seguintes objetivos
específicos:
\begin{enumerate}
\item A expansão da representação ontológica do modelo de informação AIM
3.0 para a sua versão 4.0 que é usada por ferramentas como ePAD.
\item Representar apropriadamente o criterio de classificação(\emph{staging})
do câncer TNM. Esse conhecimento tem que ser codificado em OWL e regras
SWRL.
\item Procura e Análise de dados com informações do \emph{staging} TNM,
como relatórios de diagnóstico clínico com as suas respectivas imagens
CT (foram usados os repositórios \emph{The NCI's Genomic Data Commons}
(GDC) e \emph{The Cancer Imaging Archive}).
\end{enumerate}
Com os objetivos acima traçados, demostramos que a utilização de tecnologias
da Web Semântica pode obter uma boa taxa de precisão na classificação
das lesões e, por tanto pode aumentar a qualidade e uniformidade na
interpretação de imagens por especialistas.

\section{Organização}

Este trabalho está estruturado da seguinte forma: 

Capítulo 2: Apresenta tópicos sobre Informática Biomédica e são abordadas
tecnologias de Tecnologia da Informação (TI\nomenclature{TI}{Tecnologia da Informação})
usadas nessa área que são pertinentes para infra-estrutura necessária
para este trabalho (e em outras áreas de pesquisa médica). 

Capítulo 3: Apresenta a fundamentação teórica sobre Web Semântica,
evidenciando os principais conceitos que serão utilizados no desenvolvimento
deste trabalho como ontologias e SWRL, entre outros. 

Capítulo 4: Esse capítulo apresenta os trabalhos relacionados, evidenciando
o uso de ontologias, e as principais investigações quanto ao estado
da arte relacionado ao trabalho.

Capítulo 5: Esse capítulo apresenta a metodologia utilizada para a
expansão do AIM, a codificação(SWRL e OWL) do criterio TNM, para finalmente
permitir o \emph{staging} dos pacientes como base em suas imagens. 

Capítulo 6: Este capítulo mostra os resultados obtidos 

Capítulo 7: Este capítulo conclui a dissertação de mestrado e exibe
as produções científicas obtidas com o desenvolvimento do trabalho,
as limitações encontradas e os trabalhos futuros para aperfeiçoamento.
\end{document}
